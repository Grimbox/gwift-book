\documentclass[a4paper,12pt]{book}
\usepackage[utf8]{inputenc}
\usepackage{graphicx}
\usepackage{hyperref}

\begin{document}

\author{Cédric Declerfayt, Frederick Pauchet}
\title{Minor swing with Django}
\date{2016-2032}

\frontmatter
\maketitle
\tableofcontents

\mainmatter
\part{Environnement de travail}

\chapter{Construire des applications maintenables}
\section{12 facteurs}

Pour la méthode de travail et de développement, nous allons nous baser sur les \href{https://12factor.net/fr/}{Twelve-factorApp} - ou plus simplement les 12 facteurs. 
L’idée derrière cette méthode, et indépendamment des langages de développement utilisés, consisteà suivre un ensemble de douze concepts, afin de:

\begin{enumerate}
	\item Faciliter  la  mise  en  place  de  phases  d’automatisation;  plus  concrètement,  de  faciliter  lesmises  à  jour  applicatives,  simplifier  la  gestion  de  l’hôte,  diminuer  la  divergence  entre  lesdifférents  environnements  d’exécution  et  offrir  la  possibilité  d’intégrer  le  projet  dans  unprocessus d’intégration continue ou déploiement continu
	\item Faciliter  la  mise  à  pied  de  nouveaux  développeurs  ou  de  personnes  souhaitant  rejoindrele  projet,  dans  la  mesure  où  la  mise  à  disposition  d’un  environnement  sera  grandementfacilitée
	\item Minimiser  les  divergences  entre  les  différents  environnemens  sur  lesquels  un  projetpourrait être déployé
	\item Augmenter    l’agilité    générale    du    projet,    en    permettant    une    meilleure    évolutivité architecturale et une meilleure mise à l’échelle - Vous avez 5000 utilisateurs en plus? Ajoutez un serveur et on n’en parle plus ;-)
\end{enumerate}

En pratique, les points ci-dessus permettront de monter facilement un nouvel environnement - qu’il soit sur la machine du petit nouveau dans l’équipe, sur un serveur Azure/Heroku/Digital Ocean ou votre nouveau Raspberry Pi Zéro planqué à la cave - et vous feront gagner un temps précieux. Pour reprendre de manière très brute les différentes idées derrière cette méthode, nous avons:

\begin{enumerate}
	\item ...
	\item ...
\end{enumerate}

\subsection{Une base de code unique, suivie par un système de contrôle de versions}

\chapter{Boite à outils}
\chapter{Un projet Python}
\chapter{Un projet Django}


\part{Déploiement}
\chapter{Infrastructure et composants}
\chapter{Code source}
\chapter{Supervision et mise à disposition}
\chapter{Méthodes de déploiement}
\chapter{Ressources}

\part{Backend}
\chapter{Modélisation}
\chapter{Migrations}
\chapter{Administration}
\chapter{Authentification}
\chapter{Shell}

\part{Frontend}
\chapter{Formulaires}
\chapter{Vues}
\chapter{Templates}
\chapter{URLs et espaces de noms}

\part{Concepts périphériques}
\chapter{Logging}
\chapter{Tests unitaires}

\part{Go Live}
\chapter{Gwift}

\chapter{Khana}

\chapter{Applications \textit{Legacy}}

\chapter{Refactoring}

\backmatter
% bibliography, glossary and index would go here.

\end{document}