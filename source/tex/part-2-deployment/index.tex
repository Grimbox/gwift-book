Il y a une raison très simple à aborder le déploiement dès maintenant: à trop attendre et à peaufiner son développement en local, on en oublie que sa finalité sera de se retrouver exposé sur un serveur.
Il est du coup probable d'oublier une partie des désidérata, de zapper une fonctionnalité essentielle ou simplement de passer énormément de temps à adapter les sources pour qu'elles fonctionnent sur un environnement en particulier.

Aborder le déploiement dès le début permet également de rédiger dès le début les procédures d'installation, de mises à jour et de sauvegardes.
Déploier une nouvelle version sera aussi simple que de récupérer la dernière archive depuis le dépôt, la placer dans le bon répertoire, appliquer des actions spécifiques (et souvent identiques entre deux versions), puis redémarrer les services adéquats, et la procédure complète se résumera à quelques lignes d'un script bash.

Le serveur que django met à notre disposition \textit{via} la commande `runserver` est extrêmement pratique, mais il est uniquement prévu pour la phase  développement: en production, il est inutile de passer par du code Python pour charger des fichiers statiques (feuilles de style, fichiers JavaScript, images, ...).
De même, Django propose par défaut une base de données SQLite, qui fonctionne parfaitement dès lors que l'on connait ses limites et que l'on se limite à un utilisateur à la fois. En production, il est légitime que la base de donnée soit capable de supporter plusieurs utilisateurs et connexions simultanément...
En restant avec les paramètres par défaut, il est plus que probable que vous rencontriez rapidement des erreurs de verrou parce qu'un autre processus a déjà pris la main pour écrire ses données.
En bref, vous avez quelque chose qui fonctionne, mais qui ressemble de très loin à ce dont vous aurez besoin au final.

Dans cette partie, nous aborderons les points suivants:

\begin{itemize}
    \item Définir l'infrastructure nécessaire à notre application et configurer l'hôte qui hébergera l'application: dans une machine physique, virtuelle ou dans un container. Nous aborderons aussi les déploiements via Ansible et Salt.
    \item Déployer notre code source
    \item Configurer les outils nécessaires à la bonne exécution de ce code et de ses fonctionnalités: les différentes méthodes de supervision de l'application, comment analyser les fichiers de logs, comment intercepter correctement une erreur si elle se présente et comment remonter l'information.
    \item Rendre notre application accessible depuis l'extérieur.
\end{itemize}

\chapter{Infrastructure \and composants}

Pour une mise ne production, le standard \textit{de facto} est le suivant:

\begin{itemize}
    \item Nginx comme reverse proxy
    \item HAProxy pour la distribution de charge
    \item Gunicorn ou Uvicorn comme serveur d'application
    \item Supervisor pour le monitoring
    \item PostgreSQL ou MariaDB comme base de données.
    \item Celery et RabbitMQ pour l'exécution de tâches asynchrones
    \item Redis / Memcache pour la mise à en cache (et pour les sessions ? A vérifier).
\end{itemize}
 
Si nous schématisons l'infrastructure et le chemin parcouru par une requête, nous pourrions arriver à la synthèse suivante:

\begin{itemize}
    \item L'utilisateur fait une requête via son navigateur (Firefox ou Chrome)
    \item Le navigateur envoie une requête http, sa version, un verbe (GET, POST, ...), un port et éventuellement du contenu
    \item Le firewall du serveur (Debian GNU/Linux, CentOS, ...) vérifie si la requête peut être prise en compte
    \item La requête est transmise à l'application qui écoute sur le port (probablement 80 ou 443; et \textit{a priori} Nginx)
    \item Elle est ensuite transmise par socket et est prise en compte par un des \textit{workers} (= un processus Python) instancié par Gunicorn. Si l'un de ces travailleurs venait à planter, il serait automatiquement réinstancié par Supervisord.
    \item Qui la transmet ensuite à l'un de ses \textit{workers} (= un processus Python). 
    \item Après exécution, une réponse est renvoyée à l'utilisateur.
\end{itemize}

image::images/diagrams/architecture.png[]

\section{Reverse Proxy}

Le principe du *proxy inverse* est de pouvoir rediriger du trafic entrant vers une application hébergée sur le système. 
Il serait tout à fait possible de rendre notre application directement accessible depuis l'extérieur, mais le proxy a aussi l'intérêt de pouvoir élever la sécurité du serveur (SSL) et décharger le serveur applicatif grâce à un mécanisme de cache ou en compressant certains résultats.

\section{Load balancer}


\section{Workers}


\section{Supervision}


\section{Base de données}


\section{Mise en cache}


\section{Code source}

Au niveau logiciel (la partie mise en subrillance ci-dessus), la requête arrive dans les mains du processus Python, qui doit encore

. effectuer le routage des données,
. trouver la bonne fonction à exécuter,
. récupérer les données depuis la base de données,
. effectuer le rendu ou la conversion des données,
. et renvoyer une réponse à l'utilisateur.

Comme nous l'avons vu dans la première partie, Django est un framework complet, intégrant tous les mécanismes nécessaires à la bonne évolution d'une application.
Il est possible de démarrer petit, et de suivre l'évolution des besoins en fonction de la charge estimée ou ressentie, d'ajouter un mécanisme de mise en cache, des logiciels de suivi, ...

\chapter{Outils de supervision et de mise à disposition}


\chapter{Méthode de déploiement}

Nous allons détailler ci-dessous trois méthodes de déploiement:

* Sur une machine hôte, en embarquant tous les composants sur un même serveur. Ce ne sera pas idéal, puisqu'il ne sera pas possible de configurer un \textit{load balancer}, de routeur plusieurs basées de données, mais ce sera le premier cas de figure.
* Dans des containers, avec Docker-Compose.
* Sur une *Plateforme en tant que Service* (ou plus simplement, *PaaS*), pour faire abstraction de toute la couche de configuration du serveur.

\section{Sur une machine hôte}

La première étape pour la configuration de notre hôte consiste à définir les utilisateurs et groupes de droits. Il est faut absolument éviter de faire tourner une application en tant qu'utilisateur *root*, car la moindre faille pourrait avoir des conséquences catastrophiques.

Une fois que ces utilisateurs seront configurés, nous pourrons passer à l'étape de configuration, qui consistera à:

\begin{itemize}
    \item Déployer les sources
    \item Démarrer un serveur implémentant une interface WSGI (**Web Server Gateway Interface**), qui sera chargé de créer autant de \xcancel{petits lutins} travailleurs que nous le désirerons.
    \item Démarrer un superviseur, qui se chargera de veiller à la bonne santé de nos petits travailleurs, et en créer de nouveaux s'il le juge nécessaire
    \item Configurer un proxy inverse, qui s'occupera d'envoyer les requêtes d'un utilisateur externe à la machine hôte vers notre serveur applicatif, qui la communiquera à l'un des travailleurs.
\end{itemize}

La machine hôte peut être louée chez Digital Ocean, Scaleway, OVH, Vultr, ... Il existe des dizaines d'hébergements typés VPS (**Virtual Private Server**). 
A vous de choisir celui qui vous convient \footnote{Personnellement, j'ai un petit faible pour Hetzner Cloud}.

\section{Debian}

\begin{lstlisting}[language=bash]
    apt update
    groupadd --system webapps <1>
    groupadd --system gunicorn_sockets <2>
    useradd --system --gid webapps --shell /bin/bash --home /home/gwift gwift <3>
    mkdir -p /home/gwift <4>
    chown gwift:webapps /home/gwift <5>
\end{lstlisting}
\section{Heroku}

Heroku est une Platform as a Service.

\includegraphics[scale=0.6]{deployment/heroku}
\input{part-2-deployment/docker.tex}

WARNING: le serveur de déploiement ne doit avoir qu'un accès en lecture au dépôt source.

On peut aussi passer par fabric, ansible, chef ou puppet.

\section{Supervision}

Qu'est-ce qu'on fait des logs après ? :-)

. Sentry via sentrysdk
. Nagios
. LibreNMS
. Zabbix

Il existe également https://munin-monitoring.org[Munin], https://www.elastic.co[Logstash, ElasticSearch et Kibana (ELK-Stack)] ou https://www.fluentd.org[Fluentd].

== Autres outils

Voir aussi devpi, circus, uswgi, statsd.

See https://mattsegal.dev/nginx-django-reverse-proxy-config.html

== Ressources

* https://zestedesavoir.com/tutoriels/2213/deployer-une-application-django-en-production/
* https://docs.djangoproject.com/fr/3.0/howto/deployment/[Déploiement].

* Let's Encrypt !

